\documentclass[11pt,a4paper]{article}
\usepackage{fullpage, parskip}

\begin{document}

\title{PPLS Coursework 1}
\author{Jacob Essex\\s1040340}
\date{}
\maketitle

\subsubsection*{Q1}

\begin{verbatim}
int x = 10, y = 0;
co
    while (x!=y) { ##A
        x = x-1; ##B
    }
    y = y+1; ##C
    //
    <await (x==y);> ##D
    x = 8; ##E
    y = 2; ##F
oc
\end{verbatim}

Path are represented in a simalar way to a regular expression, in that (A,B)* means 0
or more iterations of A followed by B. Where a line isn't an atomic
operation, C represents the occurence of all operations on the line. C' represents the occurence
of the first operation, C'' the second operation etc.

In many cases there are a large number of possible paths to the output.
While I have hopefully listed all possible outputs the paths given are
just an example of how a given output could be reached.

First considering only simple paths through the program that treat the
loop (A,B) and all lines as atomic we find the following ouputs and
paths:

\begin{verbatim}
x = 8, y = 3: (A,B)*, D, E, F, C
x = 8, y = 2: (A,B)*, D, E, C, F

x = 0, y = 3: (A,B)*, D, E, (A,B)*, F, C
x = 0, y = 2: (A,B)*, D, E, (A,B)*, C, F

x = 0, y = 1 (fail to terminate): (A, B)*, C, D
\end{verbatim}

The result where $x = 0$, $y = 1$ fails to terminate due to the condition
for line D not being wait. So the second block (D, E, F)

Lines C and B are clearly not atomic as they are actually both a read
and a write. Representing the operation in some hypothetical form of assembly language
may give something such as follows:

\begin{verbatim}
Load @x, r1
Add r1, 1
Store r1, @x
\end{verbatim}

Clearly other commands can be interleaved between these three commands,
leading to inconsistencies, so we cannot consider just a single thread.

The add is not observable to other processors, so in our considerations
we can reduce the complexity of our considerations by just considering
the observable events such for B and C, i.e. load and store. We labeled
these as B' and B' or C' and C'' for B and C respectively. So for B:

\begin{verbatim}
Load @x, r1 ##B'
Add r1, 1
Store r1, @x ##B''
\end{verbatim}

%This means that other operations on the variable can be interleaved
%between the read and the write. As mentioned above the notation used is
%that the operation C is in fact two operations, C and C'. Considering
%this along
Looking at paths assuming C is split into multiple operations gives the following
possible paths through the program:

\begin{verbatim}
x = 8, y = 1: (A, B)*, D, E, C', F, C''
x = 0, y = 1: (A, B)*, B, D, E, (A,B)*, C', F, C''
\end{verbatim}

There is also the case when line F sets $y$ to a value greater than $x$
causing an infinite loop. Depending on the model there is of course the
eventual result that $x$ will underflow but I assume that we are dealing with a
language that has integers that are not bounded by word length.

\begin{verbatim}
x -> -INF , y = 2: (A, B)*, D, E, (A, B)*, F, (A, B)*, C
\end{verbatim}

The following example is very dependant on the points at which the
interleaving happens. For example if at the point at the point F is
executed $x$ is less than $2$ then the program will fail to terminate with $x$
tending to minus infinity. This leads to the same set of answers as one
of the above paths but just uses a slightly more complex path to reach
the result

\begin{verbatim}
x = 2, y = 3: (A, B)*, D, E, (A, B)*, A, F, B, (A, B)*, C
\end{verbatim}

I couldn't find any cases where the lack of atomicity in lines A (due to two loads)
and B caused a difference in output.

Of course this is not the only set of paths. The code we are
considering is assumed to be unoptimized and so the code maintains its order. Due to things
such as super scaler execution and reordering, loop unrolling etc I
would suspect that the actual possible outputs are far larger and also
compiler and machine dependant. I also expect that this would be close
to impossible to analyise by hand.

\newpage
\subsubsection*{Q2}

%\begin{itemize}
%    \itemsep1pt\parskip0pt\parsep0pt
%    \item In the mp impl, messages sent are deg(v) and rndvalue(v) and the first legal colour
%    \item In the shared memory version this data is readable by everyone
%\end{itemize}

In the message passing implementation of the algorithm each processor
represents a node in the graph. Each processor holds internal state
about the node $v$ such as the list of colours it that it is impossible
for the node to be. At each iteration of the algorithm if the node
hasn't yet been allocated a colour the processor sends deg(v),
rndvalue(v) and its choice of colour to all its neighbours and preforms
analysis on that data.

This presented algorithm uses multiple iterations, where each iteration is
made up of a defined number stages. Unlike the message passing
implementation all data is readable by all nodes which means that rather
than nodes sending data to receiving nodes, the receiving nodes would
just read the data themselves. This leads to the problem that it is
possible for the receiving node to read data in an inconsistent state
and or if the data is processed at different speeds the algorithm could
move out of lockstep and so one node could be several iterations ahead
of other nodes thereby breaking the algorithm.

%\begin{itemize}
%\itemsep1pt\parskip0pt\parsep0pt
%\item This is most simalar to the Interacting Peers Pattern
%
%  \begin{itemize}
%  \item Same Program Multiple Data
%  \item Each processor responsible for maintaining data about invlaid colours etc
%  \item Use barriers to maintain syncronisation. Simalar to Jacboi algo\ldots{} mentioned in lectures
%  \item Dealing with more tasks per node just means that a single processor has to preform the work of multiple nodes per round
%  \item Sequentially
%  \item Either threads to allow each task to be run in parallel
%  \item Equlivent as long as long as the entire thing remains in lockstep
%  \end{itemize}
%\end{itemize}

\paragraph{Interacting Peers} This is most similar to the interacting peers pattern,
as each processor
has some other number of processors it interacts with defined as defined
by the graph. As with other iteration based algorithms that use this
pattern such as the Jacobi algorithm an ideal way of synchronising the
processors is to use barriers to ensure the algorithm runs in lockstep
and every processor remains at the same stage and iteration of the
algorithm as every other processor. This barrier would ensure that all
processors wait until all active processors have entered the barrier,
but given that it is possible for a node to be allocated a colour and so
not need to preform any further processing care should be taken that
only active processors need to enter the barrier.

If the algorithm was altered to allow for multiple graph nodes per
processor then for each stage of the algorithm each processor would have
to preform each the tasks for some number of nodes >= 1. 

It is possible that having multiple nodes per processor would be a significant benefit as
due to the number of barriers a significant amount of time would be spent synchronizing.
I would make the assumption that in many cases having a single processor processing 20 nodes is
faster than 20 processors each processing 1 node.

%\begin{itemize}
%\itemsep1pt\parskip0pt\parsep0pt
%\item
%  Could be implemented using bag of tasks, but doesn't fit that well
%
%  \begin{itemize}
%  \itemsep1pt\parskip0pt\parsep0pt
%  \item Would need to use a master to schedule tasks
%  \item Does have the advantage of implictly supporting more than one node per processor
%  \item Allows different performance of nodes as faster nodes just process more tasks
%  \item load distribution, deals better with nodes stopping
%  \item node not tied to a processor
%  \end{itemize}
%\end{itemize}

\paragraph{Bag of tasks} It would also be possible to implement this use the bag of tasks
pattern. This isn't quite as elegant and it would be slightly shoehorned
in however it does have several distinct advantages, most notably
the lack of a node (or set of nodes) being assigned to a single processor
which is possible due to shared memory between processors. This in turn allows
load balancing work between processors to minimize the waiting time if a
processor is idle (due to all its nodes having been assigned a colour) and is able
to deal better with a multiprogramming environment (such as the OS needing to use cycles).

Care would have to be taken to ensure that the task queue is ordered so
that the steps are processed in the correct order. Any system implanting
this would have to be taken to ensure that that tasks for a further step of the algorithm are not
started before all the tasks for the current step have finished.  The simplest method that I can see is that the bag of tasks has some understanding of the task and so requires all processors to be waiting for a task before tasks to preform the next step are made available.

Given that there is a central bag of tasks then the overhead of having a single 
processor processing for each node would be extremely inefficient as waiting for 
and getting a task could quite conceivable take longer than the task itself.

However using this method does mean that having multiple nodes per processor is trivial as the pattern doesn't make any assumption of which node belongs to which processor and so to support more nodes than processors is simply a matter of creating more tasks. This would then allow each processor to get large numbers of tasks from the bag of tasks thereby reducing the inefficacies of having a single point allocating tasks.

%\begin{itemize}
%\itemsep1pt\parskip0pt\parsep0pt
%\item In this case, neither the other two patterns metnioned in lectures makes any sense
%
%  \begin{itemize}
%  \itemsep1pt\parskip0pt\parsep0pt
%  \item Producer and consumer
%  \item Pipeline
%  \end{itemize}
%\item Not the best of ideas, maybe use a better algo? Look into this
%\end{itemize}


\paragraph{Other patterns}
Of two other patterns mentioned in the lecture, non of them make much sense to apply in this case.  

The pipeline pattern makes very lite sense given there is no linear process that the data will follow.

While each process does produce data that is sent to another process, it doesn't fit the producer consumer model as the data flow is multi-way, i.e. each producer is also a consumer. There is also no need for the standard producer consumer pattern (of for example having a circular buffer) as the algorithm works in such a way that synchronisation is achieved by the algorithm ensuring that running each processor runs each step in lockstep.


\end{document}
